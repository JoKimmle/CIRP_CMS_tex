% Template for Elsevier CRC journal article
% version 1.2 dated 09 May 2011
%------------------------------------------

%% The '3p' and 'times' class options of elsarticle are used for Elsevier CRC
%% The 'procedia' option causes ecrc to approximate to the Word template
\documentclass[5p,times,procedia]{elsarticle}
\flushbottom



%% The `ecrc' package must be called to make the CRC functionality available
\usepackage{ecrc}
\usepackage{amsmath}

% JO bibliography:
%\usepackage[sorting=none]{biblatex} %biblatex -nope!
%\addbibresource{Bibliography.bib}   %biblatex -nope!
%\usepackage{natbib}                 %natbib.  -nope!
%\bibliographystyle{unsrtnat}        %unsrtnat -nope!


%% set the volume if you know. Otherwise `00'
\volume{00}

%% set the starting page if not 1
\firstpage{1}

%% Give the name of the journal
\journalname{Procedia CIRP}

%% Give the author list to appear in the running head
%% Example \runauth{C.V. Radhakrishnan et al.}
\runauth{Author name}

%% The choice of journal logo is determined by the \jid and \jnltitlelogo commands.
%% A user-supplied logo with the name <\jid>logo.pdf will be inserted if present.
%% e.g. if \jid{yspmi} the system will look for a file yspmilogo.pdf
%% Otherwise the content of \jnltitlelogo will be set between horizontal lines as a default logo

%% Give the abbreviation of the Journal.
\jid{trpro}

%% Give a short journal name for the dummy logo (if needed)
%\jnltitlelogo{Transportation Research}

%% Hereafter the template follows `elsarticle'.
%% For more details see the existing template files elsarticle-template-harv.tex and elsarticle-template-num.tex.

%% Elsevier CRC generally uses a numbered reference style
%% For this, the conventions of elsarticle-template-num.tex should be followed (included below)
%% If using BibTeX, use the style file elsarticle-num.bst

%% End of ecrc-specific commands
%%%%%%%%%%%%%%%%%%%%%%%%%%%%%%%%%%%%%%%%%%%%%%%%%%%%%%%%%%%%%%%%%%%%%%%%%%

%% The amssymb package provides various useful mathematical symbols
\usepackage{amssymb}


% if you have landscape tables
\usepackage[figuresright]{rotating}
%\usepackage{harvard}
% put your own definitions here:x
%   \newcommand{\cZ}{\cal{Z}}
%   \newtheorem{def}{Definition}[section]
%   ...

% add words to TeX's hyphenation exception list
%\hyphenation{author another created financial paper re-commend-ed Post-Script}

% declarations for front matter

\usepackage[bookmarks=false]{hyperref}
    \hypersetup{colorlinks,
      linkcolor=blue,
      citecolor=blue,
      urlcolor=blue}


\begin{document}
\begin{frontmatter}

%% Title, authors and addresses

%% use the tnoteref command within \title for footnotes;
%% use the tnotetext command for the associated footnote;
%% use the fnref command within \author or \address for footnotes;
%% use the fntext command for the associated footnote;
%% use the corref command within \author for corresponding author footnotes;
%% use the cortext command for the associated footnote;
%% use the ead command for the email address,
%% and the form \ead[url] for the home page:
%%
%% \title{Title\tnoteref{label1}}
%% \tnotetext[label1]{}
%% \author{Name\corref{cor1}\fnref{label2}}
%% \ead{email address}
%% \ead[url]{home page}
%% \fntext[label2]{}
%% \cortext[cor1]{}
%% \address{Address\fnref{label3}}
%% \fntext[label3]{}

\dochead{57th CIRP Conference on Manufacturing Systems 2024 (CMS 2024)}%

\title{Digital Twin Based Online Material Defect Detection for CNC-Milled Workpieces}

%% use optional labels to link authors explicitly to addresses:
%% \author[label1,label2]{<author name>}
%% \address[label1]{<address>}
%% \address[label2]{<address>}



\author[a,b]{First Author} 
\author[a]{Second Author\corref{cor1}}%$\ast$}}
\author[b]{Third Author}
%\ead{author@institute.xxx}

\address[a]{Hochschule Reutlingen, ESB Business School, Alteburgstraße 150, 72762 Reutlingen, Germany}
\address[b]{Department of Industrial Engineering, Stellenbosch University, Joubert Street, Stellenbosch, 7600, South Africa}

\aucores{* Corresponding author. Tel.: +49-7121-271-5005; fax: +49-7121-271-90-5005. E-mail address: dominik.lucke@reutlingen-university.de}

\begin{abstract}
Achieving reliable lot size one compatible and adaptable online quality monitoring for CNC-milled workpieces remains elusive yet.
To address this challenge, our approach aims to bridge the current gap in research by developing a cost-effective and reference-independent monitoring concept for material defect detection in CNC-machined parts. This paper presents a novel digital twin-based method, utilizing machining vibrations and a g-code-based encoding of the cutting process. The objective is to detect material defects, such as blowholes, without the need for individual workpiece references. The proposed method aims to reduce barriers to entry, minimize waste, and enhance machine productivity by enabling automated early online quality control. To develop and validate the model, we generate a new dataset combining machining vibration with technological context data such as chip-shape. We demonstrate the feasibility and potential of the approach in a job shop setting on a 3-axis CNC mill.
\end{abstract}

\begin{keyword}
Type your keywords here, separated by semicolons ; 

%% keywords here, in the form: keyword \sep keyword

%% PACS codes here, in the form: \PACS code \sep code

%% MSC codes here, in the form: \MSC code \sep code
%% or \MSC[2008] code \sep code (2000 is the default)

\end{keyword}
\cortext[cor1]{Corresponding author. Tel.: +49-7121-271-5005; fax: +49-7121-271-90-5005. E-mail address: dominik.lucke@reutlingen-university.de}

\end{frontmatter}


%=====================================================================
%=====================================================================
\section{Introduction}\label{Sec_Introduction}


\vspace*{8pt}
\begin{nomenclature}
\begin{deflist}[A]
\defitem{A}\defterm{radius of}
\defitem{B}\defterm{position of}
\defitem{C}\defterm{further nomenclature continues down the page inside the text box}
\end{deflist}
\end{nomenclature}\vskip24pt

\subsection{Structure}

%=====================================================================
\section{Introduction}\label{Sec_Introduction}
 Moti(erste seite)
 
%=====================================================================
\section{Related work}
(state of the art) (1,5seiten)

%=====================================================================
\section{Aproach}
(vorgehen + prinzipielle aufbau/Konzept)

%=====================================================================
\section{Implementation}
(in python, auf emco, DoE...)

%=====================================================================
\section{Validation}

%=====================================================================
\section{Conclusion}
and futere recommendation (halbe seite)






Zitiertest...

\cite{Al-Naggar.Jamil.ea2021}

\cite{Axinte.Gindy2004a}

\cite{Ma.Howard.ea2020}

\cite{Biermann.Zabel.ea2013}

\cite{Chen.Lin2017}

%=====================================================================
%=====================================================================
%% For references without a BibTeX database:
% JO bibliography
%\printbibliography               %biblatex -nope!
\bibliographystyle{elsarticle-num} %elsevierspecific -yep!
\bibliography{Bibliography.bib} %natbib %elsevier is based on natbib -so yep!

\end{document}
